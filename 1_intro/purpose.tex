\subsection{既存技術の問題点}

\subsection{課題設定}
    モダリティ間の外観差は入力データ分布の差であると捉えることができる.
    ドメイン適応(Domain Adaptation, DA)は入力データの分布の差に対処するための技術であり,医療画像においてはMultimodal Segmentationなどのタスクに応用されている.
    Multimodal Registrationに対してDAを用いる手法はいくつか存在するものの,Single Step Registrationに適用されている例は非常に少ない.

\subsection{貢献}
    本論文では、DAを用いることで,Registrationの教師データを用いることなくMultimodal Registrationが可能なモデルを訓練する手法を提案する.
    本論文で我々は自然画像で訓練されたRegistration modelの性能が、DAを用いることによってどのように変化するかを観察することにより,DAの医療画像Registrationへ応用可能性を示す.